\section{The Importance of Inelastic Collisions}

\begin{frame}{The Importance of Inelastic Processes}
\begin{itemize}
	\item \alert{Interesting:} The modification of the collision integral on the RHS due to inleastic effects, leaves the \alert{time evolution of the scales invariant}!
	\item Implications on the \alert{condensate formation} can be obtained from numerical analysis of the modified transport equation.
	\item The inelastic contribution to the collision integral gives a \alert{sink term}.
	\item Balancing source (elastic) und sink (inelastic) contributions may result in a condensate surviving during most of the thermalization process.
\end{itemize}
\vspace{0.5em}
Further insights can be gained by considering e.\,g. the effect of \alert{longitudinal expansion} (cf. \cite{Blaizot2012}) or studying in more detail the effects of \alert{radiation} (cf. \cite{Blaizot2016})
\end{frame}
