\section{The Importance of Inelastic Collisions}
The treatment of the inelastic collisions will not be as detailed as for the elastic case as we will focus on the qualitative implications of allowing inelastic annihilation and creation processes during thermalization. A full treatment of all inelastic effects is nearly impossible and even today it is not yet fully understood which processes are the most relevant ones driving the system towards the equilibrium state. I will try to highlight the most important observations and refer to the respective research papers for further details. \\
\noindent 
First of all we observe that the inelastic processes will modify the collision integral on the right hand side of eqn. (\ref{eqn:transport}). Interestingly it can be shown that this modification leaves the relation for the scattering time $t$ unchanged and therefore also the evolution equation for the defining scales $\Lambda$ and $\Lambda_{\mathrm{s}}$. Implications on the condensate formation can be obtained via an extensive numerical study of solutions of the complete transport equation. The elastic contributions to the collision integral provide a source term for the condensate whereas the inelastic contributions may be interpreted as sink term. The task is to understand the balancing of the source and sink contributions which may allow a condensate to exist during most of the thermalization process.\\
\noindent
One could consider for example the effect of the strong longitudinal expansion of matter involved in relativistic heavy-ion collisions, i.\,e. the flattening of an isotropic initial distribution in the direction of the collision axis, by introducing a suitable drift term on the left hand side of the transport equation  or instabilities in the isotropy of $p_z$ and the transversal momentum $p_{\mathrm{T}}$ (cf. \cite{Blaizot2012}). Another important effect is the influence of gluon radiation allowing transport of momenta over a wide range of momenta on a very short time scale. This can be understood as a certain feature of gauge theories. For more details the interested reader is referred to publication \cite{Blaizot2016}.