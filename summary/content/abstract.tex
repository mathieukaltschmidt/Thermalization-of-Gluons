\begin{center}

	\makeatletter
	\thispagestyle{plain}
	\LARGE\textbf{\@title} \\
	\vspace{2mm}
	\large\bfseries{\@author} \\
	\normalfont
	\vspace{2mm}
	\large{\@date} \\
	\vspace{2mm}
	\large{Institute for Theoretical Physics \\
		Heidelberg University} \\
	\makeatother
\end{center}

\normalsize
\noindent This report summarizes my talk given at the Statistical Physics Seminar organized by Prof. Georg Wolschin at the Institute for Theoretical Physics in Heidelberg during the summer term 2020. \\
We study the thermalization process of gluons at the initial stages of relativistic heavy-ion collisions at energies realized for example at the RHIC at the Brookhaven National Laboratory or at the LHC at CERN. After shortly introducing the experimental and theoretical situation we  make use of kinetic theory to understand the complex thermalization process as a dynamical interplay of elastic and inelastic collisions. The second part focuses on the derivation and solution of a Nonlinear Boson Diffusion equation (NBDE) which provides a suitable and  analytically solvable model describing thermalization based on a Boltzmann-type equation.