\section{Conclusion and Outlook}
The understanding of the thermalization process of gluons plays a central role in finding an appropriate theoretical description of the complex interplay of physical processes during relativistic heavy-ion collisions. \\
\noindent
Using kinetic theory and statistical transport equations we were able to estimate important quantities such as the equilibration time $\tau_{\mathrm{eq}}$ by understanding thermalization as a dynamical interplay of elastic and inelastic scatterings. \\
\noindent
One explanation for the apparent excess of particles in the thermal spectrum compared to the expected equilibrium distribution is the formation of a Bose-Einstein condensate which may survive during most of the thermalization process. This option seems however not to be the favored interpretation amongst the researchers nowadays as the understanding of the various inelastic contributions to the gluon scattering is understood better than back when the idea of the condensate came up first.\\
\noindent
In the second part we focused on the derivation and solution of a nonlinear boson diffusion equation providing further insights into the thermalization process and an analytically accessible model to study different aspects of thermalization in more detail.\\
\noindent
In the future one may elaborate on the different approximation schemes and solution techniques for example by considering the time- and energy-dependence of the values of the transport coefficients or by extending the model in more than $1+1$ spacetime dimensions to be able to account for example for possible anisotropies.